%-----------------------------------------------------------------------------------------------------------------------------------------------%
%	The MIT License (MIT)
%
%	Copyright (c) 2021 Jitin Nair
%
%	Permission is hereby granted, free of charge, to any person obtaining a copy
%	of this software and associated documentation files (the "Software"), to deal
%	in the Software without restriction, including without limitation the rights
%	to use, copy, modify, merge, publish, distribute, sublicense, and/or sell
%	copies of the Software, and to permit persons to whom the Software is
%	furnished to do so, subject to the following conditions:
%	
%	THE SOFTWARE IS PROVIDED "AS IS", WITHOUT WARRANTY OF ANY KIND, EXPRESS OR
%	IMPLIED, INCLUDING BUT NOT LIMITED TO THE WARRANTIES OF MERCHANTABILITY,
%	FITNESS FOR A PARTICULAR PURPOSE AND NONINFRINGEMENT. IN NO EVENT SHALL THE
%	AUTHORS OR COPYRIGHT HOLDERS BE LIABLE FOR ANY CLAIM, DAMAGES OR OTHER
%	LIABILITY, WHETHER IN AN ACTION OF CONTRACT, TORT OR OTHERWISE, ARISING FROM,
%	OUT OF OR IN CONNECTION WITH THE SOFTWARE OR THE USE OR OTHER DEALINGS IN
%	THE SOFTWARE.
%	
%
%-----------------------------------------------------------------------------------------------------------------------------------------------%

%----------------------------------------------------------------------------------------
%	DOCUMENT DEFINITION
%----------------------------------------------------------------------------------------

% article class because we want to fully customize the page and not use a cv template
\documentclass[a4paper,12pt]{article}

%----------------------------------------------------------------------------------------
%	FONT
%----------------------------------------------------------------------------------------

% % fontspec allows you to use TTF/OTF fonts directly
% \usepackage{fontspec}
% \defaultfontfeatures{Ligatures=TeX}

% % modified for ShareLaTeX use
% \setmainfont[
% SmallCapsFont = Fontin-SmallCaps.otf,
% BoldFont = Fontin-Bold.otf,
% ItalicFont = Fontin-Italic.otf
% ]
% {Fontin.otf}

%----------------------------------------------------------------------------------------
%	PACKAGES
%----------------------------------------------------------------------------------------
\usepackage{url}
\usepackage{parskip} 	

%other packages for formatting
\RequirePackage{color}
\RequirePackage{graphicx}
\usepackage[usenames,dvipsnames]{xcolor}
\usepackage[scale=0.9]{geometry}

%tabularx environment
\usepackage{tabularx}

% Qty for SI units
\usepackage{siunitx}

%for lists within experience section
\usepackage{enumitem}

% centered version of 'X' col. type
\newcolumntype{C}{>{\centering\arraybackslash}X} 

%to prevent spillover of tabular into next pages
\usepackage{supertabular}
\usepackage{tabularx}
\newlength{\fullcollw}
\setlength{\fullcollw}{0.47\textwidth}

%custom \section
\usepackage{titlesec}				
\usepackage{multicol}
\usepackage{multirow}

%CV Sections inspired by: 
%http://stefano.italians.nl/archives/26
\titleformat{\section}{\Large\scshape\raggedright}{}{0em}{}[\titlerule]
\titlespacing{\section}{0pt}{10pt}{10pt}

%for publications
\usepackage[style=authoryear,sorting=ynt, maxbibnames=2]{biblatex}

%Setup hyperref package, and colours for links
\usepackage[unicode, draft=false]{hyperref}
\definecolor{linkcolour}{rgb}{0,0,0}
\hypersetup{colorlinks,breaklinks,urlcolor=linkcolour,linkcolor=linkcolour}
\addbibresource{citations.bib}
\setlength\bibitemsep{1em}

%for social icons
\usepackage{fontawesome5}

%debug page outer frames
%\usepackage{showframe}


% job listing environments
\newenvironment{jobshort}[2]
    {
    \begin{tabularx}{\linewidth}{@{}l X r@{}}
    \textbf{#1} & \hfill &  #2 \\[3.75pt]
    \end{tabularx}
    }
    {
    }

\newenvironment{joblong}[2]
    {
    \begin{tabularx}{\linewidth}{@{}l X r@{}}
    \textbf{#1} & \hfill &  #2 \\[3.75pt]
    \end{tabularx}
    \begin{minipage}[t]{\linewidth}
    \begin{itemize}[nosep,after=\strut, leftmargin=1em, itemsep=3pt,label=--]
    }
    {
    \end{itemize}
    \end{minipage}    
    }



%----------------------------------------------------------------------------------------
%	BEGIN DOCUMENT
%----------------------------------------------------------------------------------------
\begin{document}

% non-numbered pages
\pagestyle{empty} 

%----------------------------------------------------------------------------------------
%	TITLE
%----------------------------------------------------------------------------------------

% \begin{tabularx}{\linewidth}{ @{}X X@{} }
% \huge{Your Name}\vspace{2pt} & \hfill \emoji{incoming-envelope} email@email.com \\
% \raisebox{-0.05\height}\faGithub\ username \ | \
% \raisebox{-0.00\height}\faLinkedin\ username \ | \ \raisebox{-0.05\height}\faGlobe \ mysite.com  & \hfill \emoji{calling} number
% \end{tabularx}

\begin{tabularx}{\linewidth}{@{} C @{}}
\Huge{Vladimir Marinov} \\[7.5pt]
\href{}{\raisebox{-0.05\height}\faMapMarker\ London, UK} \ $|$ \ 
\href{https://linkedin.com/in/vladimir-marinov-b738611b9}{\raisebox{-0.05\height}\faLinkedin\ Vladimir Marinov} \ $|$ \ 
\href{mailto:vladi.g.marinov@gmail.com}{\raisebox{-0.05\height}\faEnvelope \ vladi.g.marinov@gmail.com} \ $|$ \ 
\href{tel:+447864894306}{\raisebox{-0.05\height}\faPhone\ +44 7864 894 306} \\
\end{tabularx}

%----------------------------------------------------------------------------------------
% EXPERIENCE SECTIONS
%----------------------------------------------------------------------------------------

%Interests/ Keywords/ Summary
\section{Profile}
I am a passionate RF and Microwave Hardware Engineer, skilled in development of electronics systems from low frequency analog to mm-Wave frequencies (\qty{110}{\giga\hertz} and beyond).

%Experience
\section{Work Experience}

\begin{joblong}{RF and Microwave Engineer - EECL}{October 2024 - Present}
    \item Led end-to-end development of a novel W-Band frequency converter (\qtyrange{70}{86}{\giga\hertz}) — from initial RF/microwave design and simulation, through manufacturing, assembly, and delivery. Published an academic paper detailing the design and results.
    \item Improved RF performance of complex microwave hardware by conducting full-wave 3D electromagnetic simulations and optimizations in Ansys HFSS.
    % \item Contributed to the design and development of ultra-high-performance RF products, including wideband frequency converters and sophisticated switch matrices, focusing on architecture definition, component selection, modeling, and performance tuning.
    \item Designed an ultra-low phase-noise LO generation subsystem to support a space-based \qtyrange{2}{18}{\giga\hertz} Software-Defined Radio (SDR).
    \item Leading ongoing technical development of \qtyrange{82}{86}{\giga\hertz} FMCW radar front-end.
\end{joblong}

\begin{joblong}{Part Time Hardware Engineer - EnduroSat}{October 2023 - April 2024}
    \item  Developed mission-critical satellite hardware in Altium (schematic and layout) with a focus on very high density PCB designs.
\end{joblong}

\begin{joblong}{Teaching Assistant - Imperial College London}{July 2022 - April 2024}
    \item  Provided teaching support for the \textit{Instrumentation} module, focusing on hardware design for precision analog measurements, and analog and high-frequency metrology and calibration.
    \item Worked on implementing a fully-functional 32-bit RISC-V processor in SystemVerilog, which is now used for teaching in the Instruction Architectures course. Provided teaching support for the course.
\end{joblong}

\begin{joblong}{Full Time Hardware Engineer - EnduroSat}{April 2023 - October 2023}
    \item  Developed satellite hardware in Altium (schematic and layout) including antennas and RF hardware, precision protected power supplies and other custom mission-specific devices.
    \item Developed internal company guidelines on EMI/EMC and grounding practices for flight hardware.
    \item Worked on reliability analysis of hardware including Failure Mode and Effect Analysis (FMEA), Fault Detection, Isolation and Recovery (FDIR), and reliability simulations in Ansys Sherlock.
\end{joblong}

\begin{joblong}{Part Time RF Engineer - EnduroSat}{July 2019 - March 2020}
    \item Worked on software and hardware realizations of Software-Defined-Radio (SDR) Ground Stations for satellite communications. 
\end{joblong}
  

%----------------------------------------------------------------------------------------
%	EDUCATION
%----------------------------------------------------------------------------------------
\section{Education}

\begin{joblong}{MEng Electrical and Electronics Engineering, Imperial College London}{2020 - 2024}
    \item Graduated with First Class Honours, and with Dean's List distinction
    \item Master's thesis on Differential Slow-Wave structures, with a focus on 3D electromagnetic simulations, RF PCB design, and precision VNA measurements of differential devices. The thesis was awarded the Nicholas Battersby prize for excellence in analog electronics.
\end{joblong}

% \begin{joblong}{Sofia High School of Mathematics, Sofia}{September 2012 - June 2020}
%    \item Graduated from one of the most prestigious high schools in Bulgaria with a program that focuses heavily on Mathematics, Physics, and Computer Science. 
% \end{joblong}


%----------------------------------------------------------------------------------------
%	PUBLICATIONS
%----------------------------------------------------------------------------------------
\section{Publications}
\textit{Design of a Highly-Integrated W-Band Up-Down Frequency Converter}   - published and presented (as first author), at Automated Radio Frequency and Microwave Measurement Society (ARMMS) Conference, April 2025 - \href{https://eecl.co.uk/wp-content/uploads/2025/05/Design-of-a-Highly-Integrated-W-Band-Up-Down-Frequency-Converter.pdf}{\faLink\ Link}

%----------------------------------------------------------------------------------------
%	PROJECTS
%----------------------------------------------------------------------------------------

\section{Projects}
\begin{joblong}{Avionics Team Lead - Karman Space Program, London}{October 2021 - July 2023}
   \item Led the avionics team of university student rocketry project, and developed custom flight computer hardware, RF telemetry hardware, and power supplies. Initiated and later support embedded software development (in C) for custom hardware. Led integration of hardware for test rocket launches and high-altitude balloon tests. 
\end{joblong}

%----------------------------------------------------------------------------------------
%	SKILLS
%----------------------------------------------------------------------------------------
\section{Skills}
\begin{tabularx}{\linewidth}{@{}l X@{}}
RF and Microwave Hardware Design &  \normalsize{RF and Microwave systems design up to and exceeding \qty{110}{\giga\hertz}}\\
PCB \& Hardware Design  &  \normalsize{Schematic Design and Layout in Altium Designer, Kicad}\\  
RF Measurement and Calibration  &  \normalsize{VNA, Spectrum Analyzers, Custom Calibration Kit Design}\\  
RF Simulation&  \normalsize{3D Electromagnetic Simulation in Ansys HFSS}\\  
FPGA&  \normalsize{RTL development in SystemVerilog, Verilog}\\  
Embedded C/C++&  \normalsize{Development of low-level firmware, driver, and RTOS}\\  
\end{tabularx}

\vfill
% \center{\footnotesize Last updated: \today}

\end{document}
